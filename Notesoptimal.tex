\documentclass[final]{siamltex}
% written by Reinhard January 23 2019
% updated by Reinhard  Feb. 1
% updated by Reinhard  Feb. 5
% updated by Reinhard March 21
% updated by Reinhard March 25 
% updated by Reinhard March 26

\usepackage{graphicx,amsmath,amsfonts,color,a4,pifont}
\usepackage{latexsym,amssymb,epsf,subfigure}
%\usepackage[hyperref=true, url=false,
%            isbn=false,
%            backref=true,
%            style=custom-numeric-comp,
%            citereset=chapter,
%            maxcitenames=3,
%            maxbibnames=100,
%            block=none]{biblatex}

\usepackage[T1]{fontenc}
\usepackage[dvips]{epsfig}
\usepackage[dvips]{graphicx}
\usepackage{pifont}
\usepackage{ifthen}
\usepackage{float} 
\usepackage[algoruled,titlenumbered]{}
\usepackage{hyperref} 
\usepackage{cite}
%\usepackage{refcheck}




\newtheorem{assumption}{Assumption}
\newtheorem{remark}{Remark}
\newtheorem{observation}{Observation}
\newtheorem{example}{Example}

%\newcommand{\mat}{\left[ \begin{array}{c}  }
%\newcommand{\rix}{\end{array} \right] }

\newcommand{\uha}{^{\frac{1}{2}}}
\newcommand{\umha}{^{-\frac{1}{2}}}

\newcommand{\cgi}{ZE^{-1}Z^T}
\newcommand{\muha}{^{-\frac{1}{2}}}
\newcommand{\innbnR}{\in\mathbb{R}^{n\times n}}
\newcommand{\innbrR}{\in\mathbb{R}^{n\times r}}
\newcommand{\innbsR}{\in\mathbb{R}^{n\times s}}
\newcommand{\innCnn}{\in\mathbb{C}^{n\times n}}
\newcommand{\innCrr}{\in\mathbb{C}^{r\times r}}
\newcommand{\innCrn}{\in\mathbb{C}^{r\times n}}
\newcommand{\innCnr}{\in\mathbb{C}^{n\times r}}
\newcommand{\innbnC}{\in\mathbb{C}^{n\times n}}
\newcommand{\innbCnnmr}{\in\mathbb{C}^{n\times n-r}}
\newcommand{\innCnmr}{\in\mathbb{C}^{n-r}}

\newcommand{\R} {\mathbb{R}}
\newcommand{\bR} {\mathbb{R}}
\newcommand{\bRnn} {\mathbb{R}^{n \times n}}

\DeclareMathOperator*{\dimn}{dim}
\DeclareMathOperator*{\rankn}{rank}
\DeclareMathOperator*{\spann}{span}


\newcommand{\padef} {P_{\mathrm{ADEF}}}
\newcommand{\U}{\mathcal{U}}


\newcommand{\Up}{\mathcal{U}^\perp}
\newcommand{\Y}{\mathcal{Y}}
\newcommand{\Z}{\mathcal{Z}}
\newcommand{\V}{\mathcal{V}}

\newcommand{\W}{\mathcal{W}}
\newcommand{\F}{\mathcal{F}}
\renewcommand{\L}{\mathcal{L}}

\newcommand{\A}{\mathbf{A}}
\newcommand{\cA}{\mathcal{A}}

\newcommand{\B}{\mathbf{B}}
\newcommand{\I}{\mathbf{I}}
\newcommand{\M}{\mathbf{M}}
\newcommand{\N}{\mathbf{N}}
\renewcommand{\S}{\mathbf{S}}
\newcommand{\D}{\mathbf{D}}
\renewcommand{\P}{\mathbf{P}}
%\newcommand{\R}{\mathbf{R}}
\newcommand{\J}{\mathbf{J}}
\renewcommand{\u}{\mathbf{u}}
\newcommand{\vc}{\mathbf{v}}
\newcommand{\f}{\mathbf{f}}
\newcommand{\Uc}{\mathbf{U}}




\newcommand{\g}{\mathbf{g}}
\newcommand{\x}{\mathbf{x}}
\newcommand{\Q}{\mathbf{Q}}
\newcommand{\y}{\mathbf{y}}
\newcommand{\w}{\mathbf{w}}
\newcommand{\Ww}{\mathbf{W}}
\renewcommand{\H}{\mathcal{H}}
\newcommand{\Hh}{\mathbf{D}}


\newcommand{\Upd}{U_{\perp}}



\newcommand{\nquad}{\mathbb{R}^{n,n}}
\newcommand{\rquad}{\mathbb{R}^{r,r}}
\newcommand{\rrec}{\mathbb{R}^{n,r}}
\newcommand{\trec}{\mathbb{R}^{n,t}}
\newcommand{\esym}{Z^TAZ}
\newcommand{\enotsym}{Y^TAZ}
\newcommand{\qnotsym}{ZE^+Y^T}
\newcommand{\indi}{\mathcal{I}}
\newcommand{\indj}{\mathcal{J}}
\newcommand{\svdy}{U_Y\Sigma_Y V_Y^T}
\newcommand{\svdz}{U_Z\Sigma_Z V_Z^T}
\newcommand{\onevec}{\mathbbm{1}}
\newcommand{\coloneq}{\mathrel{\mathop:}=}
\newcommand{\eqcolon}{=\mathrel{\mathop:}}
\newcommand{\cupdot}{\stackrel{\cdot}{\cup}}
\renewcommand{\labelenumi}{(\alph{enumi})}


\newcommand{\beqo}{\begin{eqnarray*}}
\newcommand{\beq}{\begin{eqnarray}}
\newcommand{\eeqo}{\end{eqnarray*}}
\newcommand{\eeq}{\end{eqnarray}}
%\newcommand{\bproof}{{\bf Proof: \ }}
%\newcommand{\eproof}{\hfill q.e.d. \\ }
%\newcommand{\eproof}{\hfill $\Box $ \\ }


\setlength{\parindent}{0pt}
\setlength{\parskip}{2pt}

%\onehalfspace

%\DeclareMathOperator{\im}{im}
%\DeclareMathOperator{\span}{span}





\newcommand{\matr}[2]{\left[ \begin{array}{#1} #2 \end{array} \right]}
\newcommand{\mat}{\left[ \begin{array}{c}  }
\newcommand{\rix}{\end{array} \right] }

\newcommand\norm[1]{\left\lVert#1\right\rVert}
\newcommand{\nrm}[2][]{\ensuremath{\left\|#2\right\|_{#1}}}
\newcommand{\nn}{\nonumber}



\newcommand{\Esym}{U^TAU}
\newcommand{\PDS}{I - AU(U^TAU)^{-1}U^T}


\numberwithin{equation}{section}



\newcommand{\im} {{\cal R}}
\newcommand{\kernal} {{\cal N}}
\newcommand{\nsp} {{\cal N}}
\newcommand{\ran} {{\cal R}}
\newcommand{\nul} {{\cal N}}

\newcommand{\bC}{\mathbb{C}}
\newcommand{\C}{\mathbb{C}}
\newcommand{\bCn}{\mathbb{C}^n}
\newcommand{\Kr}{\mathcal{K}}
\newcommand{\Cn}{\mathbb{C}^n}
\newcommand{\Cnn}{\mathbb{C}^{n \times n}}
\newcommand{\Crn}{\mathbb{C}^{r \times n}}
\newcommand{\Cnr}{\mathbb{C}^{n \times r}}
\newcommand{\Cnnr}{\mathbb{C}^{n \times n-r}}
\newcommand{\Crnn}{\mathbb{C}^{n-r \times n}}
\newcommand{\Cnmr}{\mathbb{C}^{n-r \times n-r}}
\newcommand{\inCnn}{\in \mathbb{C}^{n \times n}}
\newcommand{\Crr}{\mathbb{C}^{r \times r}}
\newcommand{\Projj}[2]{P_{\mathcal{#1},\mathcal{#2}}}
\newcommand{\Proj}[1]{P_{\mathcal{#1}}}





%\newcommand{\smat}[1]{\left[\begin{smallmatrix} #1\end{smallmatrix}\right]}
%\newcommand{\nquad}{\mathbb{R}^{n,n}}
%\newcommand{\rquad}{\mathbb{R}^{r,r}}
%\newcommand{\rrec}{\mathbb{R}^{n,r}}
%\newcommand{\trec}{\mathbb{R}^{n,t}}
%\newcommand{\esym}{Z^TAZ}
%\newcommand{\enotsym}{Y^TAZ}
%\newcommand{\qnotsym}{ZE^+Y^T}
%\newcommand{\indi}{\mathcal{I}}
%\newcommand{\indj}{\mathcal{J}}
%\newcommand{\svdy}{U_Y\Sigma_Y V_Y^T}
%\newcommand{\svdz}{U_Z\Sigma_Z V_Z^T}
%\newcommand{\onevec}{\mathbbm{1}}
%\newcommand{\coloneq}{\mathrel{\mathop:}=}
%\newcommand{\eqcolon}{=\mathrel{\mathop:}}
%\newcommand{\cupdot}{\stackrel{\cdot}{\cup}}
%\renewcommand{\labelenumi}{(\alph{enumi})}



\author{
Luis Garc\'{i}a Ramos\footnotemark[1]
\and
Reinhard Nabben\footnotemark[1]
}



\title{On Optimal Algebraic Multigrid Methods}


\begin{document}

\maketitle
\renewcommand{\thefootnote}{\fnsymbol{footnote}}
\footnotetext[1]{
Technische Universit\"at Berlin, Institut f\"ur Mathematik, Stra\ss e des 17.
Juni 136, D-10623 Berlin,
Germany
 (\{garcia, nabben\}@math.tu-berlin.de). 
}

\renewcommand{\thefootnote}{\arabic{footnote}}
\begin{abstract}
In this note we present an alternative way to obtain optimal
interpolation operators for two-grid methods applied to Hermitian positive
definite linear systems.  In \cite{FalVZ05,Zik08} the $A$-norm of the error
propagation operator of algebraic multigrid methods is characterized. These
results are just recently used in \cite{XuZ17, Bra18} to determine optimal
interpolation operators. Here we use a characterization not of the $A$-norm but
of the spectrum of the  error propagation operator of two-grid methods, which
was proved in  \cite{GarKN18}. This characterization holds for arbitrary
matrices. For Hermitian positive definite systems this result   leads to
optimal interpolation operators with respect to the $A$-norm in a short way, 
moreover, it also leads to optimal interpolation operators with respect to the
spectral radius. For the symmetric two-grid method (with pre- and 
post-smoothing)
the optimal interpolation operators are the same. But for a two-g
rid method with only post-smoothing the optimal interpolations (and hence the
optimal algebraic multigrid
methods) are  different.  Moreover, using the  characterization of the
spectrum,
we can show that the found  optimal interpolation operators are also optimal
with
respect to the condition number of the multigrid preconditioned system.  
\end{abstract}


\begin{keywords}
multigrid, optimal interpolation operator, two-grid methods
\end{keywords}

\begin{AMS}
65F10, 65F50, 65N22, 65N55.
\end{AMS}

\pagestyle{myheadings}
\thispagestyle{plain}
\markboth{L. Garc\'{i}a Ramos, R. Nabben} {Optimal Algebraic Multigrid}

\section{Introduction}
Typical multigrid methods to solve the linear system 
\[
Ax = b,
\]
where $A$ is an $n \times n$ matrix, consist  of two ingredients, the smoothing
and  the
coarse grid correction. The smoothing is typically done by a
few
steps of a basic stationary iterative method, like the Jacobi or Gauss-Seidel
method.  For the coarse grid correction,
 a {\it prolongation} or {\it interpolation}
operator $P \in \Cnr$ and a   {\it
restriction} operator $R \in \Crn$  are needed. The coarse grid matrix is then
defined as
\beq \label{def:multAC}
A_C :=  RAP \innCrr.
\eeq 

Here we always assume  that $A$ and $A_C$ are  non-singular. 
%Then
%let
%\beq \label{def:multQPD}
%Q := PA_C^{-1}R.
%\eeq
The multigrid or algebraic multigrid (AMG) error
propagation matrix  is then given by
 \beq \label{mgiteration}
E_M = (I-M_2^{-1}A)^{\nu_2}(I -  PA_C^{-1}RA)(I-M_1^{-1}A)^{\nu_1},
\eeq
where $M_1^{-1} \innCnn$ and $M_2^{-1} \innCnn$  are   {\it smoothers}, $\nu_1$
and $\nu_2$  are the number of pre- and post-smoothing steps respectively, and
$PA_C^{-1}R$
is
the
{\it
coarse grid
correction} matrix. The multigrid method is convergent if and only if the
spectral radius of the
error propagation matrix
$\rho(E_m)$ is less than one.
Alternatively, the norm of the error propagation matrix $\|E_M\|$ 
can be considered, where
$\|\cdot\|$
is
a
consistent matrix
norm, and in this case one has  
\[
\rho(E_M) \leq \|E_M\|.
\]  
The aim of algebraic multigrid methods is to balance the interplay between
smoothing and coarse grid correction steps. However, most of the existing AMG
methods first fix a smoother and then optimize a certain quantity to choose
the interpolation $P$ and restriction $R$.

To simplify the analysis, we assume that there exists a non-singular matrix $X$
such that
\beq \label{mgx}
(I-X^{-1}A) = (I-M_1^{-1}A)^{\nu_1}(I-M_2^{-1}A)^{\nu_2},
\eeq
it can be shown that such a matrix $X$ exists if the spectral radius of $
(I-M_1^{-1}A)^{\nu_1}(I-M_2^{-1}A)^{\nu_2}$ is less  than one, see e.g.
\cite{BenS97}. Moreover, note that the matrix $E_M$ can be
written  as 
\beq \label{mgb}
E_M = I-BA,
\eeq
where the  matrix $B$ is known as the multigrid preconditioner, i.e., $B$ is an
approximation of $A^{-1}$.
Therefore,
eigenvalue estimates of $BA$ are of interest, and  they also lead to
estimates for
the eigenvalues of
$E_M$.

%Prolongation and restriction operators are called optimal if $\rho(E_M)$ is
%minimal.

%%%%%%  Theoretical results for multigrid methods are mostly obtained for
%symmetric positive definite  matrices
%
%If the spectral radius of $E_M$ is less  than one there is a non-singular
%matrix
%$B$  such that

%Next  we will consider  the case $\nu_1 = 1$, $\nu_2 = 0$. We obtain
% \beqo 
%T = (I-M_2^{-1}A)(I - QA).
%\eeqo
%Hence, 
%\beqo
%T & = & I - M_2^{-1}A -QA  + M_2^{-1}AQA\\
%& = &  I - ( M_2^{-1}  + Q  - M_2^{-1}AQ)A \\
%& = & I - ( M_2^{-1} P_D + Q)A. 
%\eeqo
%Therefore, the matrix $B$ in \eqref{mgb} is 
%\beqo
%B =  M_2^{-1} P_D + Q,
%\eeqo
%with $P_D = I-AQ$.
%Thus $B$  is just  $P_{ADEF}$, the   adapted deflation preconditioner. 

%For  $\nu_1 = 1$ and $\nu_2 = 0$
%we have 
%\beqo
%T = (I - QA)(I-M_1^{-1}A).
%\eeqo
%Hence, 
%\beqo
%T & = & I - QA - M_1^{-1}A + QAM_1^{-1}A \\ 
%& = & I - (Q + M_1^{-1} - QAM_1^{-1})A\\
%& = & I - (Q_D M_1^{-1} + Q)A.
%\eeqo
%Thus,  the matrix $B$ in \eqref{mgb} is 
%\beqo
%B = Q_D M_1^{-1} + Q.
%\eeqo

%Now, let us consider  the general case, i.e.  $T$ is  given as in
%\eqref{mgiteration}.

%then  thereexists such amatrix  $X$, 
The following theorem, proved by Garc{\'i}a Ramos, Kehl  and Nabben in
\cite{GarKN18},
gives a characterization of the spectrum of $BA$, and hence a
characterization of the spectrum of the general error propagation matrix $E_M$.
\begin{theorem} \label{theo:mg:eig}
Let $A \innCnn$ be  non-singular, and let   $P \innCnr $ and  $R \innCrn $ such
that $RAP$ is non-singular. Moreover, let $M_1 \innCnn$ and $M_2 \innCnn $ be
such  that  that the matrices $X$ in \eqref{mgx} and $RXP$ are  non-singular.
Then the following statements hold:
\begin{enumerate}
\item[(a)] The multigrid preconditioner $B$ in
\eqref{mgb}  is non-singular. 

\item[(b)] If $\tilde P, \tilde R \in \bC^{n \times n-r}$ are matrices
such that the columns  of
$\tilde P$ and $\tilde R $ form  orthonormal  bases of $(\im (P))^\perp$ and
$(\im (R^{H}))^\perp$ respectively, then the matrices $\tilde P^HA^{-1}\tilde
R$ and $P^HX^{-1}\tilde R$
are
non-singular
and the spectrum of
$BA$  is given by
\[\sigma(BA) = \{1\} \cup \sigma(\tilde P^HX^{-1}\tilde R (\tilde
P^HA^{-1}\tilde
R)^{-1}).\]
%where the eigenvalue one  with multiplicity $r$, the other eigenvalues are
%nonzero and are   the eigenvalues of 
%\[
%\tilde P^HX^{-1}\tilde R (\tilde P^HA^{-1}\tilde R)^{-1}, 
%\]
%i.e.
%\[
%\sigma(BA) = \{1\} \cup \sigma(\tilde P^HX^{-1}\tilde R (\tilde
%P^HA^{-1}\tilde
%R)^{-1}).
%\]

\end{enumerate}
\end{theorem}

We will apply this theorem to Hermitian positive definite (HPD)  matrices to
determine
the
optimal interpolation operators of AMG methods with respect to
the
spectral radius of the error propagation matrix.
For HPD 
matrices, optimal interpolation  operators with respect to the
$A$-norm have been obtained recently in \cite{XuZ17, Bra18}.
We will show that the optimal interpolation operators with respect to the
spectral
radius
for
the
symmetric/symmetrized
multigrid
method
(with
pre-
and
post-smoothing) and
the
optimal
interpolation operator with respect to the $A$-norm are the same. But for
multigrid
with
only
a
post-smoothing step
the optimal interpolation operators with respect to the spectral radius and
$A$-norm
(and
hence
the
optimal
algebraic
multigrid
methods)
are  different. Using Theorem \ref{theo:mg:eig} we can also show that the 
interpolation operators with respect to the spectral radius are also optimal 
with
respect to the condition number of the multigrid preconditioned system. 

\section{Optimal interpolation  for Hermitian positive definite matrices}

In this section  we  consider a HPD matrix $A$. Recall that he norm induced by
$A$ (or $A$-norm) is defined for $v \in \bCn$ and $S
\in \Cnn$ by
\[
\| v \|_A^2 = (v,v)_A = \|A\uha v\|_2^2,
\]
and 
\[
\| S \|_A = \|A\uha S A\umha\|_2.
\]

We will study the  following two-grid  methods given by the error
propagation
operators
\beq \label{mge}
E_{TG} = (I-M^{-H}A)(I -  PA_C^{-1}P^HA)
\eeq
 and the symmetrized version
\beq \label{smge}
E_{STG} = (I-M^{-H}A)(I -  PA_C^{-1}P^HA)(I-M^{-1}A).
\eeq
Thus we are using $R = P^H$. The range of $P$,
i.e.
$\ran
(P)$,
is
called
the
coarse space $V_c$.
Here   we fix  the smoother $M^{-1}$ and consider $E_{TG}$ and $E_{STG}$ with
respect  to the choice of the interpolation operator $P$. So, in this note,
$E_{TG}$ and $E_{STG}$ depend on $P$. In addition, we assume that the smoother
$M^{-1}$
satisfies 
\[
 \|(I-M^{-1}A \|_A < 1,
\]
which is equivalent to the condition
\beq \label{eq:pos}
M +  M^{H} - A  \quad \mbox{is  positive definite,} 
\eeq
see, e.g., \cite{Vas08}. 

It is proved by  Falgout and Vassilevski \cite{FalV04} that
\beq \label{normeq}
\|E_{STG}\|_A = \|E_{TG}\|_A^2.
\eeq

Given a fixed  smoother $M^{-1}$ such that $\| I-M^{-1}A\|_A < 1$, many AMG
methods are designed to minimizes $ \|E_{TG}\|_A$ or a related quantity. If an
operator  $P$ minimizes    $ \|E_{TG}\|_A$  directly, $P$ is called optimal.

Zikatanov proved in \cite{Zik08} that
\[
 \|E_{TG}\|_A^2 = 1 - \frac{1}{K(V_c)},
\]
where  $ K(V_c)$ is  a  value  depending  on the  coarse  space.

% The so called XZ-identity is used to get this   result \cite{XuZ02}
Although  this 
equality is known for a long time, just  recently it is used to determine
optimal prolongation operators $P$  which lead   to a minimal  value of
$\|E_{TG}\|_A$ for a given smoother (see \cite{XuZ17, Bra18}).  Here  we give
an  alternative proof of this result using the characterization of the
eigenvalues of the multigrid iteration operator  given in Theorem
\ref{theo:mg:eig}.

But before  we consider  the  more general error propagation matrix $E_M$ in
\eqref{mgiteration} with $R= P^H$  and  $E_{M} = I - BA$.


%The error propagation matrix  $E_{M}$ can  be written  as 
%\[
%E_{M} = I - B_{M}A.
%\]

Let $\U $ be the subspace spanned  by the columns of the interpolation operator
$P$ and let $\tilde U$ be a matrix whose  columns span  $\U ^\perp$. Then
Theorem \ref{theo:mg:eig}  leads to
\[
\sigma (BA) = \{1\} \cup \sigma(\tilde U^HX^{-1}\tilde U (\tilde
U^HA^{-1}\tilde U)^{-1}).
\]
Next assume that $X$ is Hermitian positive definite and that the largest
eigenvalue of $BA$, i.e. $\lambda_{max}(BA)$,  is at most one. Then we have
$\rho(E_M) = 1 - \lambda_{min}(BA)$. In order to find an optimal interpolation
operator for the error propagation matrix  we need  to first find
\[
arg \max_{\tilde U \innbCnnmr} \min \sigma(\tilde U^HX^{-1}\tilde U (\tilde
U^HA^{-1}\tilde U)^{-1}),
\]
and then find vectors which are orthogonal to the found optimal subspace
$\mathcal{\tilde U}$.
The following Theorem  solves the   first  problem.

\begin{theorem} \label{theo:main}
Let $A, X \innCnn$ be Hermitian positive definite. Let 
\beq
\mu_1 \leq \mu_2 \leq \ldots \leq  \mu_n 
\eeq
be the  eigenvalues of the generalized eigenvalue problem $X^{-1}w = \mu
A^{-1}w$ and let $w_i$, $i = 1, \ldots, n$, be the eigenvectors corresponding
to $\mu_i$. Then
\[
\max_{\tilde U \innbCnnmr} \min \sigma(\tilde U^HX^{-1}\tilde U (\tilde
U^HA^{-1}\tilde U)^{-1}) = \mu_{r+1}
\]
which is achieved by 
\[
\tilde U = [w_{r+1}, \ldots, w_n].
\] 
\end{theorem}
\begin{proof}
Let ${\bf V}$ be the set of subspaces of $\Cnn$ of dimension $n-r$. Using the
Courant-Fischer theorem we obtain for $\tilde U \innbCnnmr$
\beqo
& &  \min \sigma(\tilde U^HX^{-1}\tilde U (\tilde U^HA^{-1}\tilde U)^{-1})\\
& = &  \min_{z \innCnmr} (z^H\tilde U^HX^{-1}\tilde Uz (z^H\tilde
U^HA^{-1}\tilde Uz)^{-1})\\
& = &  \min_{\tilde z \in {\ran (\tilde U)} } (\tilde z^HX^{-1}\tilde z (\tilde
z^HA^{-1}\tilde z)^{-1}).
\eeqo
Thus
\beqo
& &  \max_{\tilde U \innbCnnmr} \min \sigma(\tilde U^HX^{-1}\tilde U (\tilde
U^HA^{-1}\tilde U)^{-1})\\
& = & \max_{V  \in {\bf V}} \min_{\tilde z \in V} (\tilde z^HX^{-1}\tilde z
(\tilde z^HA^{-1}\tilde z)^{-1})\\
&  = & \mu_{r+1}.
\eeqo
Moreover, the matrix $\tilde U = [w_{r+1}, \ldots, w_n]$ leads  to $\mu_{r+1}$.
\end{proof}

We then have 

\begin{theorem}\label{theo:main}
Let $A \innCnn$ and $ X \innCnn$ as in \eqref{mgx} be Hermitian positive
definite. Let $
\lambda_1 \leq \lambda_2 \leq \ldots \leq  \lambda_n $
be the  eigenvalues of $X^{-1}A$  and let $u_i$, $i = 1, \ldots, n$, be the
corresponding eigenvectors. Let  $\lambda_{max}(BA) \leq 1$. Then
\beq
\min_{P} \rho(E_{M}) =  1 - \min_{P}\lambda_{min}(BA) = 1 - \lambda_{r+1}.
\eeq
An optimal interpolation operator is given by 
\[
P_{opt} = [u_{1}, \ldots , u_r].
\]
\end{theorem}
\begin{proof}
Since $\lambda_{max}(BA) \leq 1$,  we have that 
\beqo
\rho(E_{M}) =  1 - \lambda_{min}(BA).
\eeqo
Note that the eigenvalues $\lambda_i $ are  the same as the $\mu_i$ in Theorem
\ref{theo:main}.
With  Theorem  \ref{theo:main} we need  to find vectors which are orthogonal to
the eigenvectors  $w_{r+1}, \ldots , w_n$ of the generalized eigenvalue problem
$X^{-1}w = \mu A^{-1}w$. Now, consider the vectors $u_i$, $i = 1, \ldots, r$.
The vectors are also eigenvectors of the generalized eigenvalue problem  $Au =
\lambda Xu$. All  $Xu_i = w_i$  are  eigenvectors  of the generalized
eigenvalue problem $X^{-1}w = \mu A^{-1}w$. But the $w_i$ are
$X^{-1}$-orthogonal (the $X\umha w_i$ are eigenvectors of the Hermitian matrix
$X\uha A^{-1} X\uha$). Thus, the $u_i$, $i = 1, \ldots, r$ are  orthogonal to
the  $w_{r+1}, \ldots , w_n$  and $P_{opt}$   leads   to the minimal value.
\end{proof}



Now, we consider $E_{TG}$ and $E_{STG}$  defined in \eqref{mge}  and
\eqref{smge}. Again   $E_{STG}$ and $E_{TG}$    can be written   as
\beqo
E_{STG} & = & I - B_{STG}A, \\
E_{TG} & = & I - B_{TG}A,
\eeqo
for some  matrices $B_{STG}$ and $B_{TG}$ in $\Cnn$. A straightforward
computation shows  that  $B_{STG}$  is Hermitian. Lemma 2.11 of \cite{Ben01}
gives
\beq \label{ben}
\|E_{STG}\|_A = \|I - B_{STG}A\|_A = \rho(I - B_{STG}A).
\eeq
Moreover, the maximal eigenvalue of $B_{STG}A$ satisfies
$\lambda_{max}(B_{STG}A) \leq 1$, see e.g. \cite{Vas08}. We then obtain
\[
\|E_{TG}\|_A^2 = \|E_{STG}\|_A = \rho(I - B_{STG}A) = 1 -
\lambda_{min}(B_{STG}A).
\]


The matrix $X$ in \eqref{mgx} is given by 
\beq \label{defX}
X^{-1}_{STG} = M^{-H} +  M^{-1} - M^{-H} AM^{-1} = M^{-H}( M +  M^{H} -
A)M^{-1}.
\eeq
With \eqref{eq:pos} we have  that $X_{STG}$ is Hermitian positive definite.
Thus  we get

\begin{corollary} \label{coro:one}
Let  $A\inCnn$  be Hermitian positive definite. Let $ M \inCnn$ such $M + M^H -
A$ is Hermitian positive definite.
Let $X_{STG}^{-1}$  be as in \eqref{defX}.  
 Let $
\lambda_1 \leq \lambda_2 \leq \ldots \leq  \lambda_n $
be the  eigenvalues of $X_{STG}^{-1}A$  and let $v_i$, $i = 1, \ldots, n$, be
the corresponding eigenvectors. Then
\beq
\min_{P} \|E_{STG}\|_A = \min_{P}\rho(E_{STG}) =  \min_{P}\|E_{TG}\|_A^2 = 1 -
\lambda_{r+1}.
\eeq
An optimal interpolation operator is given by 
\[
P_{opt} = [v_{1}, \ldots , v_r].
\]
\end{corollary}
\begin{proof}
We have  that $X_{STG}$ is positive definite and $\lambda_{max}(B_{STG}A) \leq
1$. So Theorem \ref{theo:main} gives the desired result.
\end{proof}

Next  let us consider  the non symmetric  multigrid. 
% For the symmetric multigird  we have that the above $P_{opt}$ minimizes both
the $A$-norm and the spectral radius. Since $\sigma(B_{STG}A) \subset (0,1]$
we have   $\rho(E_{STG}) = 1 - \lambda_{min}(X_{STG}^{-1}A)$. This does not
hold for the non symmetric multigrid
We use a Hermitian positive  definite smoother $M^{-1}$. The matrix $X$ in
\eqref{mgx} is given by
\beq \label{defXtg}
X^{-1}_{TG} = M^{-1}.
\eeq

We have
\beqo
\rho(E_{TG}) = 1 - \lambda_{min}(B_{TG}A)
\ \  \mbox{or} \ \ 
\rho(E_{TG}) = -(1 - \lambda_{max}(B_{TG}A)).
\eeqo

Thus, here it is not clear, if $\lambda_{min}(B_{TG}A)$ or
$\lambda_{max}(B_{TG}A)$  leads to the spectral radius.
One way to overcome  this problem is scaling. Note that we  have for all
Hermitian positive defnite matrices $X$ and $A$ and for all matrices $\tilde U
\in \Cnnr$

%\beqo
%& &  \min \sigma(\tilde U^HX^{-1}\tilde U (\tilde U^HA^{-1}\tilde U)^{-1})\\
%& = &  \min_{z \innCnmr} (z^H\tilde U^HX^{-1}\tilde Uz (z^H\tilde
%U^HA^{-1}\tilde Uz)^{-1})\\
%& = &  \min_{\tilde z \in {\ran (\tilde U)} } (\tilde z^HX^{-1}\tilde z
%(\tilde z^HA^{-1}\tilde z)^{-1%})\\
%& \geq & \min_{\tilde z \in \bCn } (\tilde z^HX^{-1}\tilde z (\tilde
%z^HA^{-1}\tilde z)^{-1})\\
%& = & \lambda_{min}(X^{-1}A),
%\eeqo

%and similarly
\beqo
& &  \max \sigma(\tilde U^HX^{-1}\tilde U (\tilde U^HA^{-1}\tilde U)^{-1})\\
& = &  \max_{z \innCnmr} (z^H\tilde U^HX^{-1}\tilde Uz (z^H\tilde
U^HA^{-1}\tilde Uz)^{-1})\\
& = &  \max_{\tilde z \in {\ran (\tilde U)} } (\tilde z^HX^{-1}\tilde z (\tilde
z^HA^{-1}\tilde z)^{-1})\\
& \leq & \max_{\tilde z \in \bCn } (\tilde z^HX^{-1}\tilde z (\tilde
z^HA^{-1}\tilde z)^{-1})\\
& = & \lambda_{max}(X^{-1}A).
\eeqo

Hence, the  Hermitian smoother
\beqo
\hat M^{-1} = \frac{1}{\lambda_{max}(M^{-1}A)}M^{-1}
\eeqo
satisfies
\beq \label{eq:spec1}
\lambda_{max}(\hat M^{-1}A) = 1.
\eeq
With Theorem \ref{theo:mg:eig} and $X^{-1} = \hat M^{-1}$ we then have 

\beqo
\lambda_{max} ((B_{TG}A) = 1,
\eeqo
thus
\beqo
\rho(E_{TG}) = 1 - \lambda_{min } (B_{TG}A).
\eeqo

Note, that \eqref{eq:spec1} is equivalent to  $\hat M - A $ being positive
semidefinite.

Thus we  have 

%The matrix  $X^{-1}$ is just the smoother $M^{-H}$.





\begin{corollary} \label{coro:two}
Let  $A\inCnn$  be Hermitian positive definite. Let $ M \inCnn$ such $M - A$ is
Hermitian positive definite.
Let $X_{TG}^{-1} =  M^{-1}$.   
 Let $
\tilde \lambda_1 \leq \tilde \lambda_2 \leq \ldots \leq  \tilde \lambda_n $
be the  eigenvalues of $X_{TG}^{-1}A$  and let $x_i$, $i = 1, \ldots, n$, be
the corresponding eigenvectors. Then
\beq \label{eq:min.case2}
\min_{P}\rho(E_{TG}) = 1 - \tilde \lambda_{r+1}.
\eeq
An optimal interpolation operator is given by 
\beq  \label{eq:min.case2int}
P_{opt} = [x_{1}, \ldots , x_r].
\eeq
\end{corollary}
\begin{proof}
The matrix $X_{TG}^{-1} = M^{-1}$ is Hermitian positive definite. Moreover,
since $M - A$ is also Hermitian positive definite the eigenvalues of
$X_{TG}^{-1}A$ are less then  one. Thus, with Theorem \ref{theo:mg:eig},
$\lambda_{max}(B_{TG}A) = 1$.  So, with  Theorem \ref{theo:main}  we obtain
\eqref{eq:min.case2} and \eqref{eq:min.case2int}.
\end{proof}

Next let us compare the optimal interpolation with respect to the $A$-norm as
given in Corollary \ref{coro:one}, with  the optimal interpolation with respect
to the spectral radius as given in Corollary \ref{coro:two}. Using $M=M^H$ and
$M - A$ Hermitian positive definite, the vectors used in  Corollary
\ref{coro:one}
are  eigenvectors of
\beqo
X^{-1}_{STG}A = 2M^{-1}A - M^{-1}AM^{-1}A,
\eeqo
while in Corollary \ref{coro:one} we use  the eigenvectors of
\beqo
X^{-1}_{TG}A = M^{-1}A.
\eeqo
But $X^{-1}_{STG}A$ is just a polynomial in $M^{-1}A$ , where   the polynomial
is given by
\beq \label{eq:pol}
p(t) = 2t - t^2.
\eeq
Thus, the eigenvectors of both matrices are the same. Moreover, the
eigenvalues are  related
by   the above polynomial. Hence, the eigenvectors corresponding  to the
smallest eigenvalues of
$X^{-1}_{STG}A$  are the same   eigenvectors that correspond to the smallest
eigenvalues of $X^{-1}_{TG}A$.

Hence, the optimal interpolation in Corollary \ref{coro:one}  and Corollary
\ref{coro:two} are the same, if we assume that $M - A$ is hermitian positive
definite.

Next, let us have  a closer look to the non symmetric multigrid and avoid
scaling. We assume  that
the smoother $M$ is Hermitian  and  leads to a convergent scheme, i.e.
\beq  \label{eq:smoother:con:}
\rho(I - M^{-1}A) < 1, 
\eeq
which implies $\sigma(M^{-1}A) \subset (0,2).$ Thus, for the matrix $E_{TG}$
we have as above
\beqo
\rho(E_{TG}) = 1 - \lambda_{min}(B_{TG}^{-1}A) < 1 
\ \ \mbox{or} \ \
\rho(E_{TG}) = -(1 - \lambda_{max}(B_{TG}^{-1}A)) < 1.
\eeqo

Let 
\beqo
Z = \tilde U^HX_{TG}^{-1}\tilde U (\tilde U^HA^{-1}\tilde U)^{-1}).
\eeqo

Then we have $\sigma(Z) \subset (0,2)$ and with  Theorem \ref{theo:mg:eig}
\beqo
\sigma(E_{TG}) = \{0\} \cup \sigma(I-Z).
\eeqo
But $\sigma(I-Z) \subset (-1,1) $. To minimize the spectral radius of $E_{TG}$
over all interpolation we consider the matrix $(I - Z)^2$. We  obtain



\begin{theorem} \label{theo:main2}
Let  $A\inCnn$  be Hermitian positive definite. Let $ M \inCnn$ be Hermitian
such $\rho(I - M^{-1}A) < 1$.
Let $X_{TG}^{-1} =  M^{-1}$.   
 Let $
\hat \lambda_1 \leq \hat \lambda_2 \leq \ldots \leq  \hat \lambda_n $
be the  eigenvalues of $(I - X_{TG}^{-1}A)^2$  and let $y_i$, $i = 1, \ldots,
n$, be the corresponding eigenvectors. Then
\beq \label{eq:min.case3}
\min_{P}\rho(E_{TG}) = (\hat \lambda_{n-r})^{\frac{1}{2}}.
\eeq
An optimal interpolation operator is given by 
\beq  \label{eq:min.case3int}
P_{opt} = [y_{n-r+1}, \ldots , y_n].
\eeq
\end{theorem}
\begin{proof}
Using the theorem of Courant and Fischer and Theorem \ref{theo:mg:eig} we have
\beqo
& & \min_{\tilde U} \max \sigma((I - Z)^2) \\
& = & \min_{\tilde U} \max \sigma( ((\tilde U^HA^{-1}\tilde U - \tilde
U^HX_{TG}^{-1}\tilde U) (\tilde U^HA^{-1}\tilde U)^{-1})^2)\\
& = & \min_{\tilde U} \max_{z \in \bC ^{n-r}}  ((z^H(\tilde U^HA^{-1}\tilde U -
\tilde U^HX_{TG}^{-1}\tilde U)z) (z^H\tilde U^HA^{-1}\tilde Uz)^{-1})^2)\\
& = & \min_{\tilde U} \max_{y \in \ran (\tilde U)}  ((y^H(A^{-1} -
X_{TG}^{-1})y) (y^HA^{-1}y)^{-1})^2)\\
& = & \hat \lambda_{n-r}.
\eeqo
  
The optimal interpolation is then given by \eqref{eq:min.case3int}.
\end{proof}


Note, the above Theorem \ref{theo:main2} and  Corollary \ref{coro:one} lead to
clear statements. The optimal interpolation operators are given by those
eigenvectors for which the smoothing is slowest to converge.







\section{The optimal interpolation with respect to the  condition number}





Note  that for symmetric multigrid with  $M + M^H - A$ Hermitian  positive
definite  the largest eigenvalue of
$B_{STG}A$ is one (see e.g. \cite{Not15}).  As seen in the proof of Corollary
\ref{coro:two}, the same holds  for  $B_{TG}A$ when we assume  that  $M - A$ is
Hermitian  positive definite. The later
assumption can be obtained  by scaling, however, this scaling effectes the
spectral radius of the
error propagation matrix. But for the condition number of the multigrid
preconditioned system, this scaling has no effect.

Theorem \ref{theo:mg:eig} characterizes the  spectrum of $B_{STG}A$ and
$B_{TG}A$. Following the arguments above, where   we found optimal
interpolation operators, such that
$\lambda_{min}(B_{STG}A)$ and $\lambda_{min}(B_{TG}A)$ are maximal, we obtain
that the same interpolation operators are optimal with respect to the condition
number $\kappa$ of the preconditioned system. We then have

 \begin{theorem}
Let  $A\inCnn$  be Hermitian positive definite. Let $ M \inCnn$ such $M + M^H -
A$ is Hermitian positive definite.
Let $X_{STG}^{-1}$  be as in \eqref{defX}.  
 Let $
\lambda_1 \leq \lambda_2 \leq \ldots \leq  \lambda_n $
be the  eigenvalues of $X_{STG}^{-1}A$  and let $v_i$, $i = 1, \ldots, n$, be
the corresponding eigenvectors. Then
\beq
\min_P \kappa(B_{STG}A) = \frac{1}{\lambda_{r+1}}.
\eeq
An optimal interpolation operator is given by 
\[
P_{opt} = [v_{1}, \ldots , v_r].
\]
\end{theorem}
  
For the non symmetric multigrid we obtain

\begin{theorem}
Let  $A\inCnn$  be Hermitian positive definite. Let $ M \inCnn$ be Hermitian
positive definite  such that $\rho(I - M^{-1}A) < 1.$
Let $X_{TG}^{-1} =  M^{-1}$.   
 Let $
\tilde \lambda_1 \leq \tilde \lambda_2 \leq \ldots \leq  \tilde \lambda_n $
be the  eigenvalues of $X_{TG}^{-1}A$  and let $x_i$, $i = 1, \ldots, n$, be
the corresponding eigenvectors. Then
\beqo
\min_P \kappa(B_{TG}A) = \frac{1}{\tilde \lambda_{r+1}}
\eeqo
An optimal interpolation operator is given by 
\beqo
P_{opt} = [x_{1}, \ldots , x_r].
\eeqo
\end{theorem}

Note, that  in all cases of the previous sections any other interpolation
operator $\tilde P$  with  $\ran (\tilde P) = \ran (P_{opt})$ is also optimal.

\section{Conclusion}
As mentioned in \cite{XuZ17} the  $A$ in AMG methods can be seen as an $A$ for 
Abstract 
Multigrid Methods. Here  we contributed to the  theory of  
abstract multigrid methods. Based on a characterization of the spectrum of the
error propagation operator and the preconditioned system of two-grid methods
we derived optimal interpolation operators with respect  to the $A$-norm and
the spectral radius of the  error propagation operator matrix in a
short way. For the symmetric multigrid method (pre- and  post-smoothing) the
optimal interpolation operators are  the same. But for post-smoothing only
multigrid the optimal interpolations and hence the optimal algebraic multigrid
methods are different. We also showed that these interpolation operators
are optimal  with respect to the condition number of the preconditioned system.






\bibliographystyle{siamplain}
%\bibliographystyle{siam}
\bibliography{GarKN2.bib}



\end{document}
